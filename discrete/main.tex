\documentclass[journal,12pt,twocolumn]{IEEEtran}
%\usepackage{setspace}
\usepackage{gensymb}
%\usepackage{graphicx}
%\usepackage{amssymb}
%\usepackage{relsize}
\usepackage[cmex10]{amsmath}
%\usepackage{amsthm}
%\interdisplaylinepenalty=2500
%\savesymbol{iint}
%\usepackage{txfonts}
%\restoresymbol{TXF}{iint}
%\usepackage{wasysym}
\usepackage{amsthm}
%\usepackage{iithtlc}
\usepackage{mathrsfs}
\usepackage{txfonts}
\usepackage{stfloats}
\usepackage{bm}
\usepackage{cite}
\usepackage{cases}
\usepackage{subfig}
\usepackage[utf8]{inputenc}
%\usepackage{xtab}
\usepackage{longtable}
\usepackage{multirow}
%\usepackage{algorithm}
%\usepackage{algpseudocode}
\usepackage{enumitem}
\usepackage{mathtools}
\usepackage{tikz}
\usepackage{circuitikz}
\usepackage{verbatim}
%\usepackage{tfrupee}
\usepackage[breaklinks=true]{hyperref}
%\usepackage{stmaryrd}
\usepackage{tkz-euclide} % loads  TikZ and tkz-base
\usetkzobj{all}
\usepackage{listings}
    \usepackage{color}                                            %%
    \usepackage{array}                                            %%
    \usepackage{longtable}                                        %%
    \usepackage{calc}                                             %%
    \usepackage{multirow}                                         %%
    \usepackage{hhline}                                           %%
    \usepackage{ifthen}                                           %%
  %optionally (for landscape tables embedded in another document): %%
    \usepackage{lscape}     
\usepackage{multicol}
\usepackage{chngcntr}
%\usepackage{enumerate}

%\usepackage{wasysym}
%\newcounter{MYtempeqncnt}
\DeclareMathOperator*{\Res}{Res}
%\renewcommand{\baselinestretch}{2}
\renewcommand\thesection{\arabic{section}}
\renewcommand\thesubsection{\thesection.\arabic{subsection}}
\renewcommand\thesubsubsection{\thesubsection.\arabic{subsubsection}}

\renewcommand\thesectiondis{\arabic{section}}
\renewcommand\thesubsectiondis{\thesectiondis.\arabic{subsection}}
\renewcommand\thesubsubsectiondis{\thesubsectiondis.\arabic{subsubsection}}

% correct bad hyphenation here
\hyphenation{op-tical net-works semi-conduc-tor}
\def\inputGnumericTable{}                                 %%

\lstset{
%language=C,
frame=single, 
breaklines=true,
columns=fullflexible
}
%\lstset{
%language=tex,
%frame=single, 
%breaklines=true
%}

\begin{document}
%


\newtheorem{theorem}{Theorem}[section]
\newtheorem{problem}{Problem}
\newtheorem{proposition}{Proposition}[section]
\newtheorem{lemma}{Lemma}[section]
\newtheorem{corollary}[theorem]{Corollary}
\newtheorem{example}{Example}[section]
\newtheorem{definition}[problem]{Definition}
%\newtheorem{thm}{Theorem}[section] 
%\newtheorem{defn}[thm]{Definition}
%\newtheorem{algorithm}{Algorithm}[section]
%\newtheorem{cor}{Corollary}
\newcommand{\BEQA}{\begin{eqnarray}}
\newcommand{\EEQA}{\end{eqnarray}}
\newcommand{\define}{\stackrel{\triangle}{=}}

\bibliographystyle{IEEEtran}
%\bibliographystyle{ieeetr}


\providecommand{\mbf}{\mathbf}
\providecommand{\pr}[1]{\ensuremath{\Pr\left(#1\right)}}
\providecommand{\qfunc}[1]{\ensuremath{Q\left(#1\right)}}
\providecommand{\sbrak}[1]{\ensuremath{{}\left[#1\right]}}
\providecommand{\lsbrak}[1]{\ensuremath{{}\left[#1\right.}}
\providecommand{\rsbrak}[1]{\ensuremath{{}\left.#1\right]}}
\providecommand{\brak}[1]{\ensuremath{\left(#1\right)}}
\providecommand{\lbrak}[1]{\ensuremath{\left(#1\right.}}
\providecommand{\rbrak}[1]{\ensuremath{\left.#1\right)}}
\providecommand{\cbrak}[1]{\ensuremath{\left\{#1\right\}}}
\providecommand{\lcbrak}[1]{\ensuremath{\left\{#1\right.}}
\providecommand{\rcbrak}[1]{\ensuremath{\left.#1\right\}}}
\theoremstyle{remark}
\newtheorem{rem}{Remark}
\newcommand{\sgn}{\mathop{\mathrm{sgn}}}
\providecommand{\abs}[1]{\left\vert#1\right\vert}
\providecommand{\res}[1]{\Res\displaylimits_{#1}} 
\providecommand{\norm}[1]{\left\lVert#1\right\rVert}
%\providecommand{\norm}[1]{\lVert#1\rVert}
\providecommand{\mtx}[1]{\mathbf{#1}}
\providecommand{\mean}[1]{E\left[ #1 \right]}
\providecommand{\fourier}{\overset{\mathcal{F}}{ \rightleftharpoons}}
%\providecommand{\hilbert}{\overset{\mathcal{H}}{ \rightleftharpoons}}
\providecommand{\system}{\overset{\mathcal{H}}{ \longleftrightarrow}}
	%\newcommand{\solution}[2]{\textbf{Solution:}{#1}}
\newcommand{\solution}{\noindent \textbf{Solution: }}
\newcommand{\cosec}{\,\text{cosec}\,}
\providecommand{\dec}[2]{\ensuremath{\overset{#1}{\underset{#2}{\gtrless}}}}
\newcommand{\myvec}[1]{\ensuremath{\begin{pmatrix}#1\end{pmatrix}}}
\newcommand{\mydet}[1]{\ensuremath{\begin{vmatrix}#1\end{vmatrix}}}
%\numberwithin{equation}{section}
\numberwithin{equation}{subsection}
%\numberwithin{problem}{section}
%\numberwithin{definition}{section}
\makeatletter
\@addtoreset{figure}{problem}
\makeatother

\let\StandardTheFigure\thefigure
\let\vec\mathbf
%\renewcommand{\thefigure}{\theproblem.\arabic{figure}}
\renewcommand{\thefigure}{\theproblem}
%\setlist[enumerate,1]{before=\renewcommand\theequation{\theenumi.\arabic{equation}}
%\counterwithin{equation}{enumi}


%\renewcommand{\theequation}{\arabic{subsection}.\arabic{equation}}

\def\putbox#1#2#3{\makebox[0in][l]{\makebox[#1][l]{}\raisebox{\baselineskip}[0in][0in]{\raisebox{#2}[0in][0in]{#3}}}}
     \def\rightbox#1{\makebox[0in][r]{#1}}
     \def\centbox#1{\makebox[0in]{#1}}
     \def\topbox#1{\raisebox{-\baselineskip}[0in][0in]{#1}}
     \def\midbox#1{\raisebox{-0.5\baselineskip}[0in][0in]{#1}}

\vspace{3cm}

\title{
%	\logo{
Discrete: Maths Olympiad
%	}
}
\author{ G V V Sharma$^{*}$% <-this % stops a space
	\thanks{*The author is with the Department
		of Electrical Engineering, Indian Institute of Technology, Hyderabad
		502285 India e-mail:  gadepall@iith.ac.in. All content in this manual is released under GNU GPL.  Free and open source.}
	
}	
%\title{
%	\logo{Matrix Analysis through Octave}{\begin{center}\includegraphics[scale=.24]{tlc}\end{center}}{}{HAMDSP}
%}


% paper title
% can use linebreaks \\ within to get better formatting as desired
%\title{Matrix Analysis through Octave}
%
%
% author names and IEEE memberships
% note positions of commas and nonbreaking spaces ( ~ ) LaTeX will not break
% a structure at a ~ so this keeps an author's name from being broken across
% two lines.
% use \thanks{} to gain access to the first footnote area
% a separate \thanks must be used for each paragraph as LaTeX2e's \thanks
% was not built to handle multiple paragraphs
%

%\author{<-this % stops a space
%\thanks{}}
%}
% note the % following the last \IEEEmembership and also \thanks - 
% these prevent an unwanted space from occurring between the last author name
% and the end of the author line. i.e., if you had this:
% 
% \author{....lastname \thanks{...} \thanks{...} }
%                     ^------------^------------^----Do not want these spaces!
%
% a space would be appended to the last name and could cause every name on that
% line to be shifted left slightly. This is one of those "LaTeX things". For
% instance, "\textbf{A} \textbf{B}" will typeset as "A B" not "AB". To get
% "AB" then you have to do: "\textbf{A}\textbf{B}"
% \thanks is no different in this regard, so shield the last } of each \thanks
% that ends a line with a % and do not let a space in before the next \thanks.
% Spaces after \IEEEmembership other than the last one are OK (and needed) as
% you are supposed to have spaces between the names. For what it is worth,
% this is a minor point as most people would not even notice if the said evil
% space somehow managed to creep in.



% The paper headers
%\markboth{Journal of \LaTeX\ Class Files,~Vol.~6, No.~1, January~2007}%
%{Shell \MakeLowercase{\textit{et al.}}: Bare Demo of IEEEtran.cls for Journals}
% The only time the second header will appear i/year/1963s for the odd numbered pages
% after the title page when using the twoside option.
% 
% *** Note that you probably will NOT want to include the author's ***
% *** name in the headers of peer review papers.                   ***
% You can use \ifCLASSOPTIONpeerreview for conditional compilation here if
% you desire.




% If you want to put a publisher's ID mark on the page you can do it like
% this:
%\IEEEpubid{0000--0000/00\$00.00~\copyright~2007 IEEE}
% Remember, if you use this you must call \IEEEpubidadjcol in the second
% column for its text to clear the IEEEpubid ma/year/1963rk.



% make the title area
\maketitle

\newpage

%\tableofcontents

\bigskip

\renewcommand{\thefigure}{\theenumi}
\renewcommand{\thetable}{\theenumi}
%\renewcommand{\theequation}{\theenumi}

\begin{abstract}
%%\boldmath
In this letter, an algorithm for evaluating the exact analytical bit error rate  (BER)  for the piecewise linear (PL) combiner for  multiple relays is presented. Previous results were available only for upto three relays. The algorithm is unique in the sense that  the actual mathematical expressions, that are prohibitively large, need not be explicitly obtained. The diversity gain due to multiple relays is shown through plots of the analytical BER, well supported by simulations. 
%
\end{abstract}
% IEEEtran.cls defaults to using nonbold math in the Abstract.
% This preserves the distinction between vectors and scalars. However,
% if the journal you are submitting to favors bold math in the abstract,
% then you can use LaTeX's standard command \boldmath at the very start
% of the abstract to achieve this. Many IEEE journals frown on math
% in the abstract anyway.

% Note that keywords are not normally used for peerreview papers.
%\begin{IEEEkeywords}
%Cooperative diversity, decode and forward, piecewise linear
%\end{IEEEkeywords}



% For peer review papers, you can put extra information on the cover
% page as needed:
% \ifCLASSOPTIONpeerreview
% \begin{center} \bfseries EDICS Category: 3-BBND \end{center}
% \fi
%
% For peerreview papers, this IEEEtran command inserts a page break and
% creates the second title. It will be ignored for other modes.
%\IEEEpeerreviewmaketitle


%Download python codes using 
%\begin{lstlisting}
%svn co https://github.com/gadepall/school/trunk/ncert/computation/codes
%\end{lstlisting}

\renewcommand{\theequation}{\theenumi}
\begin{enumerate}[label=\arabic*.,ref=\theenumi]
%\begin{enumerate}[label=\arabic*.,ref=\thesubsection.\theenumi]
\numberwithin{equation}{enumi}
\item Let a, b, c be the sides of a triangle, and T its area. Prove: $a^2+b^2+c^2 \geq 4\sqrt{3}T$.In what case does equality hold?

 
\item Let $n > 6$ be an integer and $a_1, a_2,....... a_k$ be all the natural numbers less than n and relatively prime to n. If\\
$a_2 - a_1 = a_3 - a_2 =..........= a_k - a_{k-1} > 0$
\\prove that n must be either a prime number or a power of 2.\\

\item An infinite sequence $x_0, x_1, x_2,$....... of real numbers is said to be bounded
if there is a constant C such that $\begin{vmatrix} x_i \end{vmatrix} \leq C$ for every i $\geq 0.$ Given any real number $a > 1,$ construct a bounded infinite sequence $x_0, x_1, x_2,$.......such that\\
\\ $\begin{vmatrix} x_i - x_j \end{vmatrix} \begin{vmatrix} i - j \end{vmatrix}^a \geq 1$\\
\\for every pair of distinct non negative integers i, j.\\


 
\item Consider nine points in space, no four of which are coplanar. Each pair of points is joined by an edge (that is, a line segment) and each edge is either colored blue or red or left uncolored. Find the smallest value of n such that whenever exactly n edges are colored, the set of colored edges necessarily contains a triangle all of whose edges have the same color.\\ 

\item In the plane let C be a circle, L a line tangent to the circle C, and M a point on L. Find the locus of all points P with the following property: there exists two points Q, R on L such that M is the midpoint of QR and C is the inscribed circle of triangle PQR.


\item Let D be a point inside acute triangle ABC such that $\angle ADB = \angle ACB  + \frac{\pi}{2}$ and AC · BD = AD · BC.
\begin{enumerate}
\item Calculate the ratio (AB · CD)/(AC · BD).
\item Prove that the tangents at C to the circumcircles of $\triangle ACD$ and $ \triangle BCD$ are perpendicular.
\end{enumerate}



  
\item Let m and n be positive integers. Let $a_1$, $a_2$, ... ,$a_m$ be distinct elements of {1, 2, ... , n} such that whenever $a_i$ + $a_j$ $\leq n$ for some i, j, $1 \leq i \leq j \leq m$,
there exists k, $1 \leq k \leq m$, with $a_i$ + $a_j$ = $a_k$. Prove that\\
\\ $\frac{a_1+a_2+...+a_m}{m}$ $\geq$ $\frac{n+1}{2}$.\\

\item . Determine all ordered pairs (m, n) of positive integers such that $\frac{n^3+1}{mn-1}$ is an integer.\\

\item Show that there exists a set A of positive integers with the following property: For any infinite set S of primes there exist two positive integers $m \in A$ and $n \notin$ A each of which is a product of k distinct elements of S for some $k\geq 2$.




 

\item Determine all integers $n > 3$ for which there exist n points $A_1, ... , A_n$
in the plane, no three collinear, and real numbers $r_1, ... , r_n$ such that
for $1 \leq i < j < k \leq n$, the area of $\triangle A_iA_jA_k$ is $r_i + r_j + r_k$.\\

\item Let p be an odd prime number. How many p-element subsets A of
{1, 2, . . . 2p} are there, the sum of whose elements is divisible by p?





 
\item We are given a positive integer r and a rectangular board ABCD with
dimensions $\begin{vmatrix} AB \end{vmatrix} = 20$, $\begin{vmatrix} BC \end{vmatrix} = 12$. The rectangle is divided into a grid of 20 $\times$ 12 unit squares. The following moves are permitted on the board: one can move from one square to another only if the distance between the centers of the two squares is $\sqrt{r}$. The task is to find a sequence of moves leading from the square with A as a vertex to the square with B as a vertex.
\begin{enumerate}
\item Show that the task cannot be done if r is divisible by 2 or 3.
\item Prove that the task is possible when r = 73.
\item Can the task be done when r = 97?
\end{enumerate}

\item Let P be a point inside triangle ABC such that\\
\\ $\angle APB - \angle ACB$ = $\angle APC - \angle ABC$.\\
\\ Let D, E be the incenters of triangles APB, APC, respectively. Show that AP, BD, CE meet at a point.\\

\item Let ABCDEF be a convex hexagon such that AB is parallel to DE, BC is parallel to EF , and CD is parallel to F A. Let RA, RC, RE denote the circumradii of triangles F AB, BCD, DEF ,respectively, and let P denote the perimeter of the hexagon. Prove that\\
\\ $R_A + R_C + R_E \geq \frac{P}{2}$.


 
\item In the plane the points with integer coordinates are the vertices of unit squares. The squares are colored alternately black and white (as on a chessboard).For any pair of positive integers m and n, consider a right-angled triangle whose vertices have integer coordinates and whose legs, of lengths m and n, lie along edges of the squares.\\
Let S1 be the total area of the black part of the triangle and S2 be the total area of the white part. Let\\
\\ f(m, n) = $\begin{vmatrix} S_1 - S_2 \end{vmatrix}$.
\begin{enumerate}
\item Calculate f(m, n) for all positive integers m and n which are either both even or both odd.
\item Prove that $f(m, n) \leq \frac{1}{2} \max{m, n}$ for all m and n.
\item Show that there is no constant C such that $f(m, n) < C$ for all m and n.
\end{enumerate}

\item The angle at A is the smallest angle of triangle ABC. The points B and C divide the circumcircle of the triangle into two arcs. Let U be an interior point of the arc between B and C which does not contain A. The perpendicular bisectors of AB and AC meet the line AU at V and W , respectively. The lines BV and CW meet at T . Show that\\
\\ AU = TB + TC.


 
\item In the convex quadrilateral ABCD, the diagonals AC and BD are perpendicular and the opposite sides AB and DC are not parallel. Suppose that the point P , where the perpendicular bisectors of AB and DC meet, is inside ABCD. Prove that ABCD is a cyclic quadrilateral if and only if the triangles ABP and CDP have equal areas.\\


\item Let I be the incenter of triangle ABC. Let the incircle of ABC touch the sides BC, CA, and AB at K, L, and M, respectively. The line through B parallel to MK meets the lines LM and LK at R and S, respectively. Prove that angle RIS is acute.


 

\item Determine all finite sets S of at least three points in the plane which satisfy the following condition:
for any two distinct points A and B in S, the perpendicular bisector of the line segment AB is an axis of symmetry for S.\\

\item Let n be a fixed integer, with $n \geq 2$.\\
\begin{enumerate}
\item Determine the least constant C such that the inequality\\
\\ $\sum_{1\leq i < j\leq n} x_ix_j(x_i^2 + x_j^2) \leq C(\sum_{1 \leq i \leq n}x_i)^4$\\
\\ holds for all real numbers $x_1, ... , x_n \geq 0$\\
\item For this constant C, determine when equality holds.
\end{enumerate}

\item Determine all functions f : $R \rightarrow R$ such that\\
\\ f(x - f(y)) = f(f(y)) + xf(y) + f(x) - 1 for all real numbers x, y.



 
\item AB is tangent to the circles CAMN and NMBD. M lies between C and D on the line CD, and CD is parallel to AB. The chords NA and CM meet at P; the chords NB and MD meet at Q. The rays CA and DB meet at E. Prove that PE = QE.\\

\item $A_1A_2A_3$ is an acute-angled triangle. The foot of the altitude from $A_i$ is $K_i$ and the incircle touches the side opposite $A_i$ at $L_i$. The line $K_1K_2$ is reflected in the line $L_1L_2$. Similarly, the line $K_2K_3$ is reflected in $L_2L_3$ and $K_3K_1$ is reflected in $L_3L_1$. Show that the three new lines form a triangle with vertices on the incircle.


 


\item Twenty-one girls and twenty-one boys took part in a mathematical contest.
\begin{enumerate}
 \item Each contestant solved at most six problems.
 \item For each girl and each boy, at least one problem was solved by both of them.
\end{enumerate}
 Prove that there was a problem that was solved by at least three girls and at least three boys.\\

\item Let n be an odd integer greater than 1, and let $k_1, k_2,...., k_n$ be given integers. For each of the n! permutations a = $(a_1, a_2, …, a_n)$ of 1 , 2, …, n, let\\
\\ $S(a)=\sum_{i=1}^{n} k_ia_i$.\\
\\ Prove that there are two permutations b and c, $b \neq c$, such that n! is a divisor of S(b)-S(c).


 

\item Find all real-valued functions on the reals such that (f(x) + f(y))((f(u) + f(v)) = f(xu - yv) + f(xv + yu) for all x, y, u, v.

 
\item A convex hexagon has the property that for any pair of opposite sides the distance between their midpoints is $\sqrt{3}/2$ times the sum of their lengths Show that all the hexagon’s angles are equal.\\

\item ABCD is cyclic. The feet of the perpendicular from D to the lines AB, BC, CA are P, Q, R respectively. Show that the angle bisectors of ABC and CDA meet on the line AC if RP=RQ.

 
\item We call a positive integer alternating if every two consecutive digits in its decimal representation are of different parity. Find all positive integers n such that n has a multiple which is alternating.


 
\end{enumerate}

\end{document}


